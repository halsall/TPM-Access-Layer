\documentclass[a4paper,11pt]{article}
%\documentclass[a4paper,10pt]{scrartcl}

\usepackage[utf8]{inputenc}
\usepackage{listings}
\usepackage{color}
\usepackage{geometry}
\usepackage{url}

\geometry{a4paper, portrait, margin=1.2in}

\definecolor{mygreen}{rgb}{0,0.4,0}
\definecolor{mygray}{rgb}{0.5,0.5,0.5}
\definecolor{mymauve}{rgb}{0.58,0,0.82}

\newcommand*\justify{
  \fontdimen2\font=0.4em % interword space
  \fontdimen3\font=0.2em % interword stretch
  \fontdimen4\font=0.1em % interword shrink
  \fontdimen7\font=0.1em % extra space
  \hyphenchar\font=`\-   % allowing hyphenation
}

\title{TPM Access Layer}
\author{Alessio Magro}
\date{}

\pdfinfo{%
  /Title    (TPM Access Layer)
  /Author   (Alessio Magro)
  /Creator  ()
  /Producer ()
  /Subject  ()
  /Keywords ()
}

\begin{document}
\sloppy
\maketitle

This document serves as a rough user-guide for the {\it TPM Access Layer}
software tools. These include a C-like API for communicating with TPMs, a Python 
wrapper with extended functionality for interactive debugging and scripting, as 
well as a C++ server for communication with higher software layers. The latter 
program is primarily designed to be used by the Web-based UI application, and 
is based on a Protobuf over ZeroMQ interface. This document is not meant to 
provide a detailed technical discussion on the software's implementation. This 
access layer is still in active development, and many features are not yet 
implemented (such as loading of firmware onto the FPGAs, access to on-board SPI 
devices, and so on). This document will be continually updated as new feature 
are introduced. 

At the core of all these features lies an XML schema which describes the memory 
maps of loaded firmware and the board itself. This provides a map between 
register names or mnemonics to a memory address on the board's address space. 
In the UniBoard Control Protocol (UCP), all register and memory accesses are 
performed on a provided memory address, such that any address conversion has 
to be performed on the client side. For this reason, all checks (such as 
memory block length, bitmasks, access permissions ...) have to performed on the 
client side as well. The XML file must contain all this information. Section 
\ref{XML} describes the XML schema in detail. 

The access layer API can be used directly by custom C or C++ code. A C-like 
interface was chosen so as to make it possible to use this API from pure 
C-code, although the library itself is written in C++. The library has no 
external third-party dependencies, and is easily compilable with a modern g++ 
compiler (currently version 4.8+). Section \ref{TPMAPI} provides additional 
details on this layer. However, it is generally tedious to debug and test 
functionality using compiled C code, and for this reason a Python wrapper is 
also provided. This loads the compiled C++ library and provides direct access to 
the underlying API. The wrapper extends the functionality of the underlying 
library by providing Pythonesque behavior. Section \ref{Python} describes this 
wrapper in further detail.

\section{XML Memory Map}
\label{XML}

The XML file provides a mapping between register and memory block 
names/mnemonics to memory address on the board. In the TPM's case, the XML file 
can contain the mapping for up to three 'devices': the CPLD (or board), the 
firmware loaded on the first FPGA and the firmware loaded on the second FPGA. 
The same firmware can be loaded on both FPGAs. The register in a firmware map 
can be logically split into components (such as channeliser, beamformer, 
calibrator ...), and in turn each register can be composed of multiple 
bit-fields, where disjoint sets of bits have different interpretation in the 
firmware. For this reason, an XML mapping file is composed of a hierarchy of 
four nodes (excluding the root node):
\begin{description}
 \item[Device node] Highest level node, representing the board itself (CPLD) or 
the firmware loaded on the FPGAs. The identifiers for nodes in this level are 
``CPLD'', ``FPGA1'' and ``FPGA2''.
 \item[Component node] Registers for each device can be grouped into 
components, which makes it easier to logically partition the functionality of 
these registers. For example, registers belonging to different firmware 
components can be grouped into different nodes at this level. A base address can 
be specified for each component.
 \item[Register node] Each component can be composed of multiple registers or 
memory blocks, each requiring an identifier (the name or mnemonic) and the 
memory-mapped address. Additional attributes can also be specified, such as the 
size in words, access permission, the bitmask and a description. 
 \item[Bit node] A register can be composed of a combination of multiple 
bitfields. These bitfields are specified in the fourth node level, having an 
entry for each bitfield in the register. A bitmask per bitfield must be 
specified.
\end{description}

The XML listing below provides a very concise example of an XML mapping file. 
Each of the node attributes are defined as follows:

\begin{description}
 \item[id] The device, register, memory block or bitfield name/mnemonic. This 
does not need to be unique within the XML file (such that the same firmware can 
be loaded on both FPGAs). The IDs of successive levels in the hierarchy are 
combined to generate the full name in the access layer. For example, the 
full name of the first bit in \texttt{bit\_register}, \texttt{regfile}, 
\texttt{FPGA1} in the example below would be 
\texttt{\justify regfile.bit\_register.bit1} (the registers are partitioned 
internally by device). 

 \item[address] The memory address to which the register is mapped. A base 
address can be specified at all levels (except bitfields). In this case, 
successive base addresses are added together to form the final address. 

 \item[permission] Access permissions for the register or memory block. Allowed 
entries are \texttt{r}, \texttt{w} and \texttt{rw}.

 \item[size] The size in words of the register or memory block. If this value 
is not specified, then the register is assumed to be of size 1 (32-bits). 
Memory blocks can essentially be defined by setting this value. Note that if a 
bitmask is specified for a memory block with size greater than 1, then it is 
applied for all consecutive words.
 
 \item[mask] The bitmask to be applied to the values after reading from or 
before writing to the memory address. 

 \item[description] A textual description for the register.
\end{description}

\lstset{
  language=XML,
  captionpos=b,              
  commentstyle=\color{mygreen},    
  stringstyle=\color{mygreen},
  keywordstyle=\color{blue},
  morekeywords={node,id,address,encoding,mask,permission,size,description}
}

\begin{lstlisting}[caption=XML memory map]
<?xml version="1.0" encoding="ISO-8859-1" ?>

<node>

  <node id="CPLD">
    <node id="regfile" address="0x00000000">
      <node id="register1" address="0x00000010" permission="rw" 
            size="1" mask="0xFFFFFFFF" description="..."/>
      ...
    </node>
    ...
  </node>
  
  <node id="FPGA1" address="0x100000000">
    <node id="regfile" address="0x00001000">
      <node id="bit_register" address="0x00000010" />
        <node id="bit1" mask="0x00000001" permission="rw" 
              description="..." />
        <node id="bit2" mask="0x00000002" permission="rw" 
              description="..." />
        ...
      </node>
      ...
    </node> 
    ...
  </node>
  
  <node id="FPGA2"> ... </node>
  
</node>
\end{lstlisting}

\section{TPM Access Layer API}
\label{TPMAPI}

The TPM Access Layer provides a C-like API for communicating with TPMs using 
UCP. Although monitoring and control of TPMs is the primary use-case for this 
API, its software design allows the same layer to be usable with other FPGA 
boards. Minimal changes are required to make it compatible with UniBoards 
(since they use the same protocol), whilst custom Protocol and Board classes 
are required to make it compatible with other boards. Details on how to change 
and extend this layer will be provided in a separate technical guide. The 
listing in Appendix B lists the currently defined API calls in this layer. The 
access layer is capable of handling connections to multiple boards, which could 
be of different types. 


\section{Python wrapper}
\label{Python}

The Python wrapper can be used for interactive testing and scripting. All the 
functionality is defined in the package \texttt{tpm}. This section will provide 
a rough tutorial on how to use this package. In the following examples, 
the wrapper will be connecting to one TPM board and issue requests via the 
underlying access layer. Before these interactions can occur, two operations 
need to be performed: connect to the board and load the XML memory map. The 
listing below shows how to do this.

\lstset{ %
  backgroundcolor=\color{white},   
  basicstyle=\footnotesize,        
  breakatwhitespace=false,         
  breaklines=true,                 
  captionpos=b,                    
  commentstyle=\color{mygreen},    
  deletekeywords={...},            
  escapeinside={\%*}{*)},          
  extendedchars=true,              
  frame=none,                    
  keepspaces=true,                 
  keywordstyle=\color{blue},       
  language=Python,                       
  numbers=none,                    
  numbersep=5pt,                   
  numberstyle=\tiny\color{mymauve}, 
  rulecolor=\color{black},         
  showspaces=false,                
  showstringspaces=false,          
  showtabs=false,                  
  stepnumber=2,                    
  stringstyle=\color{mymauve},     
  tabsize=2,
  title=\lstname                   
}

\begin{lstlisting}
 from tpm import *
 
 # Create a new TPM instance
 tpm = TPM(ip="127.0.0.1", port=10000)
 
 # Call load firmware to load XML file
 tpm.loadFirmwareBlocking(Device.FPGA_1, "/path/to/xml/file")
\end{lstlisting}

The first line imports all definitions from the \texttt{tpm} package. The next 
statement then creates a new instance of class \texttt{TPM}. When an IP and 
port are specified, as in this example, the \texttt{connectBoard} in the 
library is called, setting up all required internal 
data structures. It will also check to see whether the destination IP and port 
can process UCP requests, and if not, an error is generated. For this reason, 
either a TPM is required, or a script which acts as a UCP server. You can get 
this script from the \texttt{scripts} directory in the code repository, called 
\texttt{mock\_tpm.py}. Just run this on a separate console, and it will bind 
itself to all local IP address and wait for read and write requests. The 
\texttt{TPM} 
constructor also accepts an additional argument, the filepath of the compiled 
access layer, in cases where the library is not directly accessible by the 
wrapper. This can be passed with \texttt{library='path'}. 

Once connected, the XML memory file needs to be loaded. The way in which this 
will be performed on the board is not yet defined, so for the time being the 
\texttt{loadFirmwareBlocking} call must be used. This takes the device on which 
the ``firmware'' will be loaded and the path to the XML file. \texttt{Device} 
is a representation of the \texttt{DEVICE} enum in the library, and can 
currently have three values: \texttt{Device.FPGA\_1}, \texttt{Device.FPGA\_2} 
and \texttt{Device.Board}. Only the two former cases are allows for this 
functional call, since you cannot load a firmware on the board itself. This 
call will also populate all internal data structures containing register and 
memory block mapping information, such that read and write operations can 
immediately be issued by the user.

The wrapper provides two ways in which registers can be read from or written 
to, one based on direct function calls and the other using indexed access. The 
following listing provides the function definitions for using the functions 
directly:

\begin{lstlisting}
 # Read register values from specified register
 # device   - Device to which this register belongs to
 # register - Register name/mnemonic
 # n        - Number of words to read
 # offset   - Address offset to start reading from
 # Returns: Value or list of values
 def readRegister(device, register, n = 1, offset = 0)
 
 # Write values to specified register
 # device   - Device to which this register belongs to
 # register - Register name/mnemonic
 # values   - Value or list of values to write
 # offset   - Address offset to start reading from
 # Return:  Success or Failure
 def writeRegister(device, register, values, offset = 0)
 
 # Read values from specified memory address
 # address  - Memory address
 # n        - Number of words to read
 # Returns: Value or list of values
 def readAddress(address, n = 1)
 
 # Read register values from specified register
 # address  - Memory address
 # values   - Value or list of values to write
 # Returns: Success or Failure
 def writeAddress(address, values):
\end{lstlisting}

The first two functions defined above require both the \texttt{device} and 
\texttt{register} name to be able to compute the memory address. In all cases, 
checks are performed by the wrapper to make sure that requests don't go out of 
bounds (for example, the length of the list of values to write are larger than 
the size of the register, the register has write permissions in case of writes, 
and so on). The list of registers which have been loaded can be displayed either 
with the \texttt{listRegisterNames()} function call, or with the following 
statement: \texttt{print tpm}, where tpm is a \texttt{TPM} class instance. This 
will print out the list of register with additional information, as shown below:

\begin{lstlisting}
 Device  Register                           Address        Bitmask
------  --------                           -------        -------
FPGA 1	regfile.block2048b                 0x1000L        0xFFFFFFFF
FPGA 1	regfile.date_code                  0x0L           0xFFFFFFFF
FPGA 1	regfile.debug                      0x10L          0xFFFFFFFF
FPGA 1	regfile.jesd_channel_disable       0xcL           0x0000FFFF
FPGA 1	regfile.jesd_ctrl.bit_per_sample   0x4L           0x000000F0
FPGA 1	regfile.jesd_ctrl.debug_en         0x4L           0x00000001
FPGA 1	regfile.jesd_ctrl.ext_trig_en      0x4L           0x00000002
FPGA 1	regfile.reset.global_rst           0x8L           0x00000002
FPGA 1	regfile.reset.jesd_master_rstn     0x8L           0x00000001
Board   regfile.ada_ctrl.ada_a_ada4961     0x30000010L    0x00000008
Board   regfile.ada_ctrl.ada_fa_ada4961    0x30000010L    0x00000008
Board   regfile.ada_ctrl.ada_latch_ada4961 0x30000010L    0x00000008
... (redacted)
\end{lstlisting}

Indexed access provides a more Pythonesque methods of accessing registers. The 
three statements below will perform the same operation. All of them will read 
all the values for memory block \texttt{regfile.block2048b}, which is defined 
for \texttt{FPGA1}. The first statement specifies the full register name as 
provided by the access layer. In cases where the same full name is specified 
for both FPGAs (same firmware is loaded) the device name can also be specified 
as in the second example. In the third statement, the memory address is 
provided immediately, such that the \texttt{readAddress} function is called 
instead of \texttt{readRegister}.

\begin{lstlisting} 
 1. tpm['regfile.block2048b']
 
 2. tpm['fpga1.regfile.block2048b']
 
 3. tpm[0x1000]
\end{lstlisting}

Similar statement can also be defined for writing values. In the examples 
below, the memory region defined for regfile.block2048b (512 words) will be 
initialized with \texttt{1} for all values. Note that in order to write to an 
address offset, the function calls must be used directly.

\begin{lstlisting}
 1. tpm['regfile.block2048b'] = [1] * 512
 
 2. tpm['fpga1.regfile.block2048b'] = [1] * 512
 
 3. tpm[0x1000] = [1] * 512
\end{lstlisting}

The wrapper and underlying library also handle bitmasks automatically, as shown 
in the example below, where \texttt{regfile.bitregister.bit8} at address 
\texttt{0x10001000} has a bitmask of \texttt{0x00000008}

\begin{lstlisting}
 1. print tpm['regfile.bitregister.bit8']      Output = 0

 2. tpm['regfile.bitregister.bit8'] = 1
 
 3. print tpm['regfile.bitregister.bit8']      Output = 1
    
 4. print tpm[0x10001000]                      Output = 8
\end{lstlisting}

Reads from and writes to a bitfield will automatically be shifted such that the 
value belonging to the bitfield itself are displayed. Writes to a bitfield are 
performed as a read-modify-write operation in the access layer library. The 
Python wrapper also provides some additional features, as show below:

\begin{lstlisting}
 
            # Use regular experssions to search for a particular register 
            # from list of register, returning a dictionary for each 
            # match. Can also print out the results
 Statement: tpm.findRegister("*block2048b*", display = True)
 Output:    regfile.block2048b:
            ------------------
            Address:       0x1000L
            Type:          RegisterType.FirmwareRegister
            Device:        Device.FPGA_1
            Permission:    Permission.ReadWrite
            Bitmask:       0xFFFFFFFF
            Bits:          32
            Size:          512
            Description:   For testing
            
            # Return number of registers
 Statement: len(tpm)
 Output:    32
 
\end{lstlisting}

Once all operations have been performed on a TPM, the the script should 
disconnect from the board:

\begin{lstlisting}
 tpm.disconnect()
\end{lstlisting}


\pagebreak

\section*{Appendix A - Compilation and Installation}

The Access Layer source code can be retrieved from 
\url{https://github.com/lessju/TPM-Access-Layer.git}. The top directory 
contains four directories: 
\begin{itemize}
 \item \texttt{doc}, which contains some documents, example XML files and this 
user guide
 \item \texttt{python}, which contains the Python wrapper and setup scrip
 \item \texttt{script}, which contains helper scripts
 \item \texttt{src}, which contains the source code for the library and server
\end{itemize}

The \texttt{src} directory in turn contains three directories, one containing 
the source code for the C++ library (\texttt{library}), one containing the 
source code of the server (\texttt{server}) and one containing source code for 
exporting the library as a Windows DLL (experimental, \texttt{windows}). Each 
directory contains a Makefile which can be used to build and install each tool. 

\subsection*{Compiling the library}

Requirements: g++ version 4.0+. Compilation and installation steps:
\begin{enumerate}
 \item \texttt{export ACCESS\_LAYER=/path/to/top/level/access/layer/dir}
 \item \texttt{cd \$ACCESS\_LAYER/src/library}
 \item Edit \texttt{INSTALL\_DIR}, \texttt{LIBRARY\_NAME} and \texttt{GCC} in 
the Makefile so that it can be compiled and installed on your system. Note that 
\texttt{GCC} should point to a C++ compiler
 \item \texttt{make}
 \item \texttt{sudo make install}
\end{enumerate}

\subsection*{Installing the Python Wrapper}

Ideally, the python version should be 2.5 or higher. Older versions can also be 
used, however the \texttt{ctypes} package will need to be installed. 

\begin{enumerate}
 \item \texttt{sudo pip install enum34} (Assuming pip is installed)
 \item \texttt{export ACCESS\_LAYER=/path/to/top/level/access/layer/dir}
 \item \texttt{cd \$ACCESS\_LAYER/python}
 \item \texttt{sudo python setup.py install}
\end{enumerate}


\subsection*{Compiling the server}

The server uses two third party libraries: ZeroMQ and Protobuf (latest versions 
of both). You will need to install these in order to use the server.

\begin{enumerate}
 \item \texttt{export ACCESS\_LAYER=/path/to/top/level/access/layer/dir}
 \item \texttt{cd \$ACCESS\_LAYER/src/server}
 \item \texttt{protoc --cpp\_out=. message.proto}. This only needs to be 
performed once, or when message.proto changes
 \item Edit \texttt{LIBRARY\_DIR} and \texttt{GCC} in the Makefile so that it 
can be compiled on your system.
 \item \texttt{make}
\end{enumerate}

If the access layer library was not installed in a system directory, 
then update \texttt{LD\_LIBRARY\_PATH} to point to the library, otherwise the 
server won't manage to load the library: \texttt{export 
LD\_LIBRARY\_PATH=\$LD\_LIBRARY\_PATH:/path/to/library}.

\pagebreak

\section*{Appendix B - Access Layer API}
\label{API}

\lstset{ %
  backgroundcolor=\color{white},   
  basicstyle=\footnotesize,        
  breakatwhitespace=false,         
  breaklines=true,                 
  captionpos=b,                    
  commentstyle=\color{mygreen},    
  deletekeywords={...},            
  escapeinside={\%*}{*)},          
  extendedchars=true,              
  frame=none,                    
  keepspaces=true,                 
  keywordstyle=\color{blue},       
  language=C,                 
  otherkeywords={*,ID, RETURN, VALUES, UINT, STATUS, 
REGISTER,_INFO,REGISTER,DEVICE,...},         
  numbers=none,                    
  numbersep=5pt,                   
  numberstyle=\tiny\color{mymauve}, 
  rulecolor=\color{black},         
  showspaces=false,                
  showstringspaces=false,          
  showtabs=false,                  
  stepnumber=2,                    
  stringstyle=\color{mymauve},     
  tabsize=2,
  title=\lstname                   
}

\begin{lstlisting}[caption=TPM Access Layer API listing]
// Connect to a board with specified IP and port. Returns unique
// board identifier
ID connectBoard(const char* IP, unsigned short port);

// Disconnect from board with ID id
RETURN disconnectBoard(ID id);

// Reset board (currently not implemented)
RETURN resetBoard(ID id);

// Get board status (currently not implemented)
STATUS getStatus(ID id);

// Get a list of board and firmware registers. A memory map needs 
// to be loaded.
REGISTER_INFO* getRegisterList(ID id, UINT *num_registers);

// Read register value from specified device/register. Number of 
// values and address offset can be specified.
VALUES readRegister(ID id, DEVICE device, REGISTER reg, UINT n, 
                    UINT offset = 0);

// Write register value to specified device/register. Number of 
// values and address offset can be specified.
RETURN writeRegister(ID id, DEVICE device, REGISTER reg, UINT n, 
                     UINT *values, UINT offset = 0);

// Read a number of values from address. To be used for debug mode. 
VALUES readAddress(ID id, UINT address, UINT n);

// Write a number of values to address. To be used for debug mode. 
RETURN writeAddress(ID id, UINT address, UINT n, UINT *values);

// Load (non-blocking) firwmare to device. Behaviour currently unspecified
RETURN loadFirmware(ID id, DEVICE device, const char* bitstream);

// Load (blocking) firmware to device. Behaviour currently unspecified.
RETURN loadFirmwareBlocking(ID id, DEVICE device, const char* bitstream);
\end{lstlisting}

\texttt{ID}, %\texttt{RETURN}, \texttt{STATYS}, \texttt{REGISTER_INFO}, 
\texttt{VALUES}, \texttt{UINT} and \texttt{DEVICE} are defined in {\it 
Definitions.hpp}. This header file also contain a macro to enable or disable 
INFO output during execution.

\end{document}
